\usepackage{savesym}
\savesymbol{iiint}
\savesymbol{iint}
\usepackage[toc,acronym]{glossaries}
\restoresymbol{TXF}{iiint}
\restoresymbol{TXF}{iint}
\usepackage{xparse}

% makeindex.exe -s Methodologie.ist -t Methodologie.glg -o Methodologie.gls Methodologie.glo
% makeindex.exe -s Methodologie.ist -t Methodologie.alg -o Methodologie.acr Methodologie.acn
% https://en.wikibooks.org/wiki/LaTeX/Glossary#Defining_acronyms
\DeclareDocumentCommand{\newdualentry}{ O{} O{} m m m m } {
  \newglossaryentry{gls-#3}{name={#4},text={#4\glsadd{#3}},
    description={#6},#1
  }
  \makeglossaries
  \newacronym[see={[Glossaire:]{gls-#3}},#2]{#3}{#4}{#5\glsadd{gls-#3}}
}

%Général
\newacronym{URL}{URL}{Uniform Ressource Locator}
\newacronym{REGEX}{REGEX}{Regular Expression}

%Langages
\newacronym{T-SQL}{T-SQL}{Transact-SQL}
\newacronym{J2EE}{Java EE}{Java Enterprise Edition}

%Sécurité web
\newacronym{XSS}{XSS}{Cross-Site Scripting}
\newacronym{CSRF}{CSRF}{Cross-Site Request Forgery}
\newacronym{CORS}{CORS}{Cross-Origin Ressource Sharing}
\newacronym{HTTP}{HTTP}{HyperText Transfer Protocol}
\newacronym{HTTPS}{HTTPS}{HyperText Transfer Protocol Secure}
\newacronym{DNS}{DNS}{Domain Name System}
\newacronym{CMS}{CMS}{Content Management System}
\newacronym{SQL}{SQL}{Structured Query Language}
\newacronym{ORM}{ORM}{Object-Relational Mapping}
\newacronym{EDM}{EDM}{Entity Data Model}
\newglossaryentry{dorks}
{
  name={dorks},
  description={Commande spécifique à un moteur de recherche permettant d'obtenir des informations appartenant au \textit{Deep Web}, c'est-à-dire non visibles aisément par la navigation standard: adresse IP, fichiers PDF...}
}

%Architecture
\newacronym{SGDB}{SGDB}{Système de Gestion de Base de données}
\newacronym{ACL}{ACL}{Access Control List}
\newacronym{DMZ}{DMZ}{DeMilitarized Zone}
\newacronym{IDS}{IDS}{Intrusion Detection System}
\newacronym{IPS}{IPS}{Intrusion Prevention System}
\newacronym{WAF}{WAF}{Web Application Firewall}
\newacronym{NAT}{NAT}{Network Address Translationl}
\newglossaryentry{WiFi}
{
  name={Wi-Fi},
  description={Norme de transmission basée sur la IEEE 802.11b et décrivant une utilisation du \gls{WLAN}}
}
\newdualentry{WLAN}{WLAN}{Wireless LAN}{Utilisation d'onde radio pour connecter les terminaux d'un réseau, par abus de langage désigné sous le terme de \gls{WiFi}}
\newdualentry{VLAN}{VLAN}{Virtual LAN}{Technologie permettant de créer des sous-réseau au sein d'un même \gls{LAN} ou d'assurer une qualité de service. Cette technologie est basée sur la norme 802.1Q}
\newdualentry{LAN}{LAN}{Local Area Network}{Système d'information limité à un lieu géographique en opposition à \gls{WAN}} 
\newacronym{WAN}{WAN}{Wide Area Network}

%Configuration
\newacronym{JSON}{JSON}{JavaScript Object Notation}
\newacronym{RBAC}{RBAC}{Role-Based Acces Control}

%Cryptographie
%%algorithmes
\newacronym{RSA}{RSA}{Rivest Shamir Adleman}
\newacronym{DES}{DES}{Data Encryption Standard}
\newacronym{3DES}{3DES}{Triple Data Encryption Algorithm}
\newacronym{AES}{AES}{Advanced Encryption Standard}
\newacronym{SHA}{SHA}{Secure Hash Algorithm}

%%mode de chiffrement
\newacronym{ECB}{ECB}{Electronic CodeBook}
\newacronym{CBC}{CBC}{Cipher Block Chaining}
\newacronym{PCBC}{PCBC}{Propagating Cipher Block Chaining/Plaintext-Cipher Block Chaining}
\newacronym{CFB}{CFB}{Cipher FeedBack}
\newacronym{OFB}{OFB}{Output FeedBack}
\newacronym{CTR}{CTR}{Counter Mode}
\newacronym{GCM}{GCM}{Galois-Counter Mode}
\newacronym{CCM}{CCM}{Counter with CBC-MAC}

%%Institut et autorité
\newdualentry{NIST} % label
  {NIST}            % abbreviation
  {National Institute of Standards and Technology}  % long form
  {Institut gouvernemental américain adressant les recommandations de normes concernant l'informatique (voir aussi \gls{FIPS}, \gls{IETF}, \gls{IEEE} et \gls{RFC}) } % description
\newdualentry{IETF}{IETF}{Internet Engineering Task Force}{Groupe informel définissant les standards d'Internet à l'aide de \gls{RFC} (voir aussi \gls{IEEE})} 
\newdualentry{IEEE}{IEEE}{Institute of Electrical and Electronics Engineers}{Association d'informatique normalisant notamment les réseaux à travers les normes 802.X mais aussi POSIX}
\newdualentry{RFC}{RFC}{Request for Comments}{Descriptions des aspects techniques des technologies d'Internet. Il ne s'agit pas toujours de normes à part entière mais de référentiels} 
\newdualentry{FIPS}{FIPS}{Federal Information Processing Standards}{Standards définis par le \gls{NIST} décrivant notamment l'usage d'AES, DES ou les courbes élliptiques}
\newdualentry{SECG}{SECG}{Standards for Efficient Cryptography Group}{Consortium pour la standardisation de la cryptographie fondé par Certicom en 1998 incluant notamment le \gls{NIST} et Visa}
\newdualentry{NSA}{NSA}{National Security Agency}{Organisation gouvernementale américaine pour la collecte de renseignements à l'étranger et la sécurisation des communications gouvernementales. La NSA fournit notamment des recommandations en cryptographie (le CNSA) pour les contractuels du gouvernement américain}
\newdualentry{ANSSI}{ANSSI}{Agence Nationale de la Sécurité des Systèmes d'Information}{Service du gouvernement français dépendant du Ministère de la Défense et se positionnant en autorité sur la sécurité des Systèmes d'Informations de la France}

%% Concept mathématiques
\newglossaryentry{ZnZ}
{
  name={\ensuremath{\mathbb{Z}/n\mathbb{Z}}},
  description={Groupe fini cyclique d'ordre n, dans le contexte de la cryptographie, n est le produit de deux premiers. On utilise $\mathbb{Z}/p\mathbb{Z}$ pour les groupes cycliques premiers, il s'agit alors d'un corps premier fini},
  sort=ZnZ
}

%% Protocole et Famille
\newacronym{DH}{DH}{Diffie-Hellman}
\newacronym{DHE}{DHE}{Diffie-Hellman Ephemeral}
\newacronym{PQC}{PQC}{Post-Quantum Cryptography}
\newacronym{ECDH}{ECDH}{Eliptic Curve Diffie-Hellman}
\newacronym{ECC}{ECC}{Eliptic Curve Cryptography}
\newacronym{SSL}{SSL}{Secure Sockets Layer}
\newacronym{TLS}{TLS}{Transport Layer Security}


\glsresetall
\makeglossaries