%Recommandations de l'ANSSI
% http://www.ssi.gouv.fr/administration/guide/cryptographie-les-regles-du-rgs/
\newcommand{\refAnnexeB}{\footnote{2015: \url{http://www.ssi.gouv.fr/uploads/2015/01/RGS_v-2-0_B1.pdf}}}

%ANSSI
%% ZnZ (applicable RSA et DH)
\newcommand{\ZnZnAnssi}{3072\refAnnexeB}
\newcommand{\ZnZeAnssi}{$2^{16}$ bits\refAnnexeB}
\newcommand{\ZnZdAnssi}{3072\refAnnexeB\footnote{\`A partir de 2030, 2048 bits est suffisant à ce jour}}
\newcommand{\DHrecosizeAnssi}{\ZnZnAnssi}

%% ECC et ECDH
\newcommand{\ECDHrecosizeAnssi}{P-256\refAnnexeB}

%% Chiffrement par bloc
\newcommand{\blockrecosizeAnssi}{128\refAnnexeB\footnote{\`A partir de 2020, 64 bits est suffisant à ce jour}}
\newcommand{\solidityrecosizeAnssi}{128\refAnnexeB\footnote{\`A partir de 2020, 100 bits est suffisant à ce jour}}

%ECRYPT II & ENISA






% EPC
%http://www.europeanpaymentscouncil.eu/index.cfm/knowledge-bank/epc-documents/guidelines-on-cryptographic-algorithms-usage-and-key-management/epc342-08-v5-0-guidelines-on-cryptographic-algorithms-usage-and-key-management2/
\newcommand{\EPCarraysecuritybits}{
\begin{tabular}{| p{1cm} | p{3cm} |  p{4cm} |  p{3cm} | p{1cm} |}
\hline
$n$\footnote{$n$ bits de sécurité signifie qu'un attaquant aura besoin de $2^n$ opérations (une opération équivalent au temps de chiffrement) pour casser la sécurité} & Algorithme de chiffrement symétrique & Algorithme de condensat & RSA (ou tout algorithme basé sur $\mathbb{Z}/n\mathbb{Z}$) & ECC \\\hline
80 & 2DES & SHA1 & 1024 & 160\\\hline
112 & 3DES & SHA224 et SHA$3_{224}$ & 2048 & 224\\\hline
128 & AES128 & SHA256 et SHA$3_{256}$ & 3072 & 256\\\hline
192 & AES192 & SHA384 et SHA$3_{384}$ & 7680 & 384\\\hline
256 & AES256 & SHA512 et SHA$3_{512}$ & 15360 & 512\\\hline
\end{tabular}
}




%NIST
%http://csrc.nist.gov/publications/PubsSPs.html
%% Diffie-Hellman
\newcommand{\DHrecosizeNIST}{3072\footnote{2013: \url{http://nvlpubs.nist.gov/nistpubs/SpecialPublications/NIST.SP.800-56Ar2.pdf}}\footnote{\`A partir de 2030, 2048 bits est suffisant à ce jour}}
\newcommand{\ECDHrecosizeNIST}{P-512\footnote{2013: \url{http://nvlpubs.nist.gov/nistpubs/SpecialPublications/NIST.SP.800-56Ar2.pdf}}}

%% ECC
\newcommand{\ECCNIST}{$\mathbb{F}_p$: P-192, P-224, P-256, P-384, P-521; $\mathbb{F}_{2^m}$: K-163, B-163, K-233, B-233, K-283, B-283, K-409, B-409, K-571, B-571 \footnote{2013: Annexe B \url{http://nvlpubs.nist.gov/nistpubs/FIPS/NIST.FIPS.186-4.pdf}}}
\newcommand{\ECCrecoNIST}{P-192, P-224, P-256, P-384, P-521\footnote{2013: Annexe B: \url{http://nvlpubs.nist.gov/nistpubs/FIPS/NIST.FIPS.186-4.pdf}}}

% CNSA
%Recommandation de la NSA Suite B TS -
%https://www.iad.gov/iad/library/ia-guidance/ia-solutions-for-classified/algorithm-guidance/assets/public/upload/Commercial-National-Security-Algorithm-CNSA-Suite-Factsheet.pdf
%https://www.iad.gov/iad/programs/iad-initiatives/cnsa-suite.cfm
%% Diffie-Hellman
\newcommand{\DHrecosizeCNSA}{3072\footnote{2016: \url{https://www.iad.gov/iad/programs/iad-initiatives/cnsa-suite.cfm} basée sur la RFC 3526}}
\newcommand{\ECDHrecosizeCNSA}{P-384\footnote{2016: \url{https://www.iad.gov/iad/programs/iad-initiatives/cnsa-suite.cfm} basée sur la RFC 3526}}


% General
%%SEC2-v2 ECC
\newcommand{\ECCSECG}{$\mathbb{F}_p$: secp192, secp224, secp256, secp384, secp521; $\mathbb{F}_{2^m}$: sect163, sect233, sect239,sect283, sect409, sect571\footnote{2010: \url{http://www.secg.org/sec2-v2.pdf}}}

%%Brainpool Standard https://tools.ietf.org/html/rfc5639
\newcommand{\ECCBP}{brainpoolP160, brainpoolP192, brainpoolP224, brainpoolP256, brainpoolP320, brainpoolP384, brainpoolP512\footnote{2010: \url{https://tools.ietf.org/html/rfc5639}}}
%% Curve25519


% Recommandation de l'auteur
\newcommand{\DHminsize}{2048}
\newcommand{\DHrecosize}{4096}
\newcommand{\blockrecosize}{128}
\newcommand{\blockminsize}{128}

